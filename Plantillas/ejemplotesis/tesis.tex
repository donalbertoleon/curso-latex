\documentclass[12pt,letterpaper,oneside]{report}
\usepackage[utf8]{inputenc}
\usepackage[spanish]{babel}
%\usepackage{geometry}  
\usepackage{fancyhdr}
\pagestyle{fancy}
\lhead[]{\leftmark}
\rhead[]{Juana López}
\chead[]{}
\cfoot[\thepage]{\thepage}
\lfoot[]{}
\rfoot[]{}
%\renewcommand{\chaptermark}[1]{%
%	\markboth{#1}{}} %% Para quitar numeración en las cabeceras
\usepackage{graphicx}
\title{El uso de \LaTeX en las ciencias sociales y las humanidades}
\author{Juana López}
\date{} %%% Se deja en blanco para que no aparezca la fecha
\usepackage{lipsum} % Para agregar contenido vacío
\usepackage[colorlinks=true]{hyperref} %%% Para los links, las opciones son para que no estorben
\usepackage{footnotebackref} % Hace reversible la nota
\usepackage{epigraph} %%% Epígrafe
\setlength\epigraphrule{0pt} %%% Borrar línea de epígrafe
%%%%%%%%%%%
\begin{document}


	\begin{titlepage}
		\centering
		{\includegraphics[width=0.2\textwidth]{escudo}\par}
		\vspace{1cm}
		{\bfseries\LARGE Instituto de Investigaciones\\Dr. José María Luis Mora \par}
		\vspace{1cm}
		{\scshape\LARGE El uso de \LaTeX {} en las ciencias sociales y las humanidades. \par}
		\vspace{2cm}
		\makebox[8cm][s]{\Huge T E S I S}\\[8pt]
	QUE PARA OBTENER EL TÍTULO DE:\\[5pt]
	{\bfseries\Large Licenciada en Historia \par}
	\vspace{2cm}
		{\Large Presenta: \par}
		{\Large \textbf{Juana López} \par}
		\vfill
		{Asesora: \par}
		{ Miranda Juárez \par}
		\vfill
		
		{Septiembre 2021 \par}
		\end{titlepage}

\thispagestyle{empty}
\vspace*{\fill}
\epigraph{\dots si yo fuera Maradona, frente a cualquier portería, si yo fuera Maradona nunca me equivocaría\dots\\ La vida es una tómbola de noche y de día\dots}{\textemdash Manu Chao, \textit{La vida tómbola}}


\thispagestyle{empty}


\chapter*{Agradecimientos}
Gracias a mi paciencia y a mi interés, porque estuve muchas veces a punto de estudiar matemáticas para aprender \LaTeX.


\tableofcontents

\chapter{Introducción}

\lipsum[1-4]

\section{Sección}

\lipsum[1-3]

\begin{quote}
	\lipsum[1] (Mengano, 2000, p. 2)
\end{quote}

\subsection{Subsección}

\lipsum[4-6]


\chapter{Desarrollo}


\lipsum[1-4]

\begin{figure}[ht!]
	\label{img1}
	\caption{Ñu meditando}
	\centering	\includegraphics[width=.8\textwidth]{linuximagen}
\end{figure}

\section{Desarrollo 1}

\lipsum[1-2]

\begin{quote}
	\lipsum[1] (Fulano, 1985, p. 365)
\end{quote}

\section{Desarrollo 2}

\lipsum[2-4]


\chapter{Otra introducción}

\lipsum[1-4]

\section{Otra sección}

\lipsum[1-3]

\subsection{Otra subsección}

\lipsum[4-6]


\chapter{conclusión}

\lipsum[1-4]

\chapter*{Bibliografía}

Apellido, N. (año). \emph{Título del libro}. País: Editorial.

Apellido, N. (año). \emph{Título del libro}. País: Editorial.

Apellido, N. (año). \emph{Título del libro}. País: Editorial.

Apellido, N. (año). \emph{Título del libro}. País: Editorial.

\clearpage{\pagestyle{empty}\cleardoublepage}

\section*{Otras fuentes}

\subsection*{Archivos}

Apellido, N. Nombre del documento. Año. Clasificación. Archivo, País.

Apellido, N. Nombre del documento. Año. Clasificación. Archivo, País.

Apellido, N. Nombre del documento. Año. Clasificación. Archivo, País.

\section*{Material visual}

Autor, N. (año). \emph{Obra} [Formato]. País.

Autor, N. (año). \emph{Obra} [Formato]. País.

Autor, N. (año). \emph{Obra} [Formato]. País.

Autor, N. (año). \emph{Obra} [Formato]. País.



\appendix

\chapter{La importancia del apéndice}

\lipsum[1-10]


\end{document}
