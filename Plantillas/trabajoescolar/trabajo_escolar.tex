\documentclass[12pt,letterpaper,oneside]{article}
\usepackage[utf8]{inputenc}
\usepackage[spanish]{babel}
\usepackage[top=3cm, bottom=3cm, left=3cm, right=3cm,]{geometry} %% Margen
\usepackage{setspace} %%% Para interlineado
\usepackage{fancyhdr}
\pagestyle{fancy}
\lhead[]{Mi trabajo de unidad}
\rhead[]{Ramírez León}
\chead[]{}
\cfoot[]{}
\lfoot[]{\emph{Historia Social de las Revoluciones}}
\rfoot[]{\thepage}
%%%%% Eeste estilo es para definir una página sin los encabezados que se determinan en el el estilo fancy global del documento
\fancypagestyle{plain}{
	\setlength{\topmargin}{-20.4mm}
	\fancyhead[L]{}
	\fancyhead[C]{}
	\fancyhead[R]{}
	\fancyfoot[L]{}
	\fancyfoot[C]{\thepage}
	\fancyfoot[R]{}
	\renewcommand{\headrulewidth}{0pt}
	\renewcommand{\footrulewidth}{0pt}
}

\usepackage{authblk}
\usepackage{graphicx}
\usepackage[colorlinks=true]{hyperref}
\usepackage{footnotebackref}
\usepackage{lipsum} %% Para generar texto ciego
\usepackage{hanging} %% Indentar textos
\begin{document}

\thispagestyle{plain} %%%% Estilo de página plano
\begin{flushleft}
	{\bfseries Instituto de Investigaciones Dr. José María Luis Mora} \\ 
	{\itshape Maestría en Historia Contemporánea}\\ \underline{Materia: Historia Social de las Revoluciones}\end{flushleft} %%%% Encabezado

\begin{flushright}\vspace{-15mm} %%% Espacio vertical
	\includegraphics[scale=1]{escudo} %%%%% Escudo institucional
\end{flushright}

\begin{center}\vspace{1cm}
	\textbf{\LARGE Mi trabajo de unidad}\\
	\vspace{1em}   %TITULO
	Mario Alberto Ramírez León\\                         %NOMBRE
\end{center}

\begin{abstract}
  \noindent \lipsum[1]
\end{abstract}

\onehalfspace %% Interlineado a 1.5

\section*{Introducción}

\noindent \lipsum[1-3]

\section*{Desarrollo}

\noindent \lipsum[1-2]\footnote{Nota a Pie. \lipsum[1]}

\begin{figure}[ht]
	\caption{El Ñu que medita}
	\includegraphics[scale=.35]{linuximagen}\\ %%% Doble barra invertida para marcar salto de línea
	\footnotesize \center Fuente: Fulana de Tal (2021, p. 12).
	
\end{figure}

\subsection*{Subsección}

\noindent \lipsum[1-2]

\begin{quotation}
	\lipsum[1-2] (Mengana, 2020, p. 15-16).
\end{quotation}

\section*{Conclusiones}

\noindent \lipsum[1]

\section*{Referencias}

\begin{hangparas}{6.2mm}{1}
	Abreu, A. A. de, Beloch, I., Lattman-Weltman, F. e Lamarão, S. (2001).	\emph{Dicionário histórico-biográfico brasileiro, pós-1930} (2. ed.~rev.e atualizada). Brasil: Fundação Getúlio Vargas/Centro de Pesquisa e	Documentação de História Contemporânea do Brasil.
	
	Abreu, M. de P. (Org.). (1990). \emph{A Ordem do progresso: Cem anos de	política econômica republicana, 1889-1989}. Rio de Janeiro: Editora Campus.
	
	Agência Nacional do Petróleo, Gás Natural e Biocombustíveis (2015).	\emph{Petróleo e Estado}. Brasil: Autor. 
	
	Alves, J. A. B. (2020). \emph{Impactos dos Royalties do Pré-Sal no Pré-Sal no Desenvolvimento dos Municípios Costeiros do Sudeste	Brasileiro} (Tese de Doutorado). Universidade do Vale do Itajaí, Brasil.
	
	Andrade, R. B. de. (2011). A regulação do pós-lavra no direito minerário brasileiro. \emph{Revista de Direito, Estado e Recursos Naturais},	\emph{1}(1), 79--106.
	
\end{hangparas}

\end{document}