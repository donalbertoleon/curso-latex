\documentclass[12pt,letterpaper,oneside]{article}
\usepackage[utf8]{inputenc}
\usepackage[spanish]{babel}
\usepackage[margin=3cm]{geometry} 
\usepackage{fancyhdr}
\pagestyle{fancy}
\lhead[]{Artículo académico 1}
\rhead[]{\author{López, Sánchez y Martínez}}
\chead[]{}
\cfoot[\thepage]{\thepage}
\lfoot[]{}
\rfoot[]{\emph{En evaluación}}
\usepackage{authblk}
\usepackage{graphicx}
\usepackage[colorlinks=true]{hyperref}
\usepackage{footnotebackref}
\title{Artículo académico 1}
\author[a]{Juan López\thanks{juan@hotmail.com}}
\author[b]{Margarita Sánchez\thanks{margarita@hotmail.com}}
\author[a,b]{Liliana Martínez\thanks{liliana@hotmail.com}}
\affil[a]{Universidad Autónoma de México, Campus I: Iztapalapa, México.}
\affil[b]{Colegio de la Frontera Norte, Tijuana, México.}
\renewcommand\Authand{ y } %% Secparador de coautor
\renewcommand\Authands{ y } %% Separador de autores
\renewcommand*{\Affilfont}{\small\it} %% Da formato a las instituciones
\usepackage{lipsum} %% Para generar texto ciego
\usepackage{hanging}
\begin{document}
\maketitle
\begin{abstract}
	\lipsum[1]
\end{abstract}

\section{Introducción}

\lipsum[1-4]

\subsection{Subsección}

\lipsum[1-3]

\begin{figure}[ht!]
	\label{img1}
	\caption{Ñu meditando}
\centering	\includegraphics[width=.8\textwidth]{linuximagen}
\end{figure}

\subsubsection{Subsubsección}

\lipsum[4-5]

\section{Desarrollo}

\lipsum[1-2]\footnote{Esta es una nota a pie.}

\section{Conclusiones}

\lipsum[1-2]

\section{Referencias}

\begin{hangparas}{6.2mm}{1}
	Abreu, A. A. de, Beloch, I., Lattman-Weltman, F. e Lamarão, S. (2001).	\emph{Dicionário histórico-biográfico brasileiro, pós-1930} (2. ed.~rev.e atualizada). Brasil: Fundação Getúlio Vargas/Centro de Pesquisa e	Documentação de História Contemporânea do Brasil.
	
	Abreu, M. de P. (Org.). (1990). \emph{A Ordem do progresso: Cem anos de	política econômica republicana, 1889-1989}. Rio de Janeiro: Editora Campus.
	
	Agência Nacional do Petróleo, Gás Natural e Biocombustíveis (2015).	\emph{Petróleo e Estado}. Brasil: Autor. 
	
	Alves, J. A. B. (2020). \emph{Impactos dos Royalties do Pré-Sal no Pré-Sal no Desenvolvimento dos Municípios Costeiros do Sudeste	Brasileiro} (Tese de Doutorado). Universidade do Vale do Itajaí, Brasil.
	
	Andrade, R. B. de. (2011). A regulação do pós-lavra no direito minerário brasileiro. \emph{Revista de Direito, Estado e Recursos Naturais},	\emph{1}(1), 79--106.
	
\end{hangparas}

\end{document}