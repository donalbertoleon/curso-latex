\documentclass[12pt,c5paper,oneside]{book} %% Creo que el mora utiliza C5 de papel
\usepackage[utf8]{inputenc}
\usepackage[spanish]{babel}
\usepackage[margin=3cm]{geometry}  
\usepackage{fancyhdr}
\pagestyle{fancy}
\lhead[]{\leftmark}
\rhead[]{Juan López}
\chead[]{}
\cfoot[\thepage]{\thepage}
\lfoot[]{}
\rfoot[]{}

\renewcommand{\chaptermark}[1]{%
	\markboth{#1}{}} %% Para quitar numeración en las cabeceras

\usepackage{authblk}
\usepackage{graphicx}
\title{Libro académico 1: \LaTeX en las ciencias sociales y las humanidades}
\author{Juan López}
\date{} %%% Se deja en blanco para que no aparezca la fecha
\usepackage{lipsum} % Para agregar contenido vacío
\usepackage{pdfpages} % Para agregar la portada se necesita este paquete
\usepackage[colorlinks=true]{hyperref} %%% Para los links, las opciones son para que no estorben
\usepackage{footnotebackref} % Hace reversible la nota
%%%%%%%%%%%%%%%%%%%%%%
%%%%% Inicia Cuerpo%%%
%%%%%%%%%%%%%%%%%%%%%%
\begin{document}
\includepdf[pages=1]{cover} %%% Se agrega portada del libro en un archivo expterno PDF

 {\let\cleardoublepage\relax \frontmatter} %%% Elimina la página extra que llegase a existir

\maketitle

\clearpage{\pagestyle{empty}\cleardoublepage}%% Elimina pies y cabezas en las hojas vacías después de los elementos.


\begin{titlepage}
	\centering
	\vspace{1cm}
	\textsc{\Large Editorial} \par
	\vspace{3cm}
	{\scshape\Huge Libro académico 1: \LaTeX { }en las ciencias sociales y las humanidades \par}
	\vspace{3cm}
	\vfill
	{\Large \emph{Juan López} \par}
	\vfill
	{\Large Septiembre 2021 \par}
\end{titlepage} 
\clearpage{\pagestyle{empty}\cleardoublepage}

\frontmatter
\tableofcontents

\chapter{Prólogo}

\lipsum[1-7]

\chapter{Notas preeliminares}

\lipsum[1-7]

\mainmatter

\part{El comienzo}

\chapter{Introducción}

\lipsum[1-4]

\section{Sección}

\lipsum[1-3]

\begin{quote}
	\lipsum[1] (Mengano, 2000, p. 2)
\end{quote}

\subsection{Subsección}

\lipsum[4-6]

\clearpage{\pagestyle{empty}\cleardoublepage}

\chapter{Desarrollo}


\lipsum[1-4]

\begin{figure}[ht!]
	\label{img1}
	\caption{Ñu meditando}
	\centering	\includegraphics[width=.8\textwidth]{linuximagen}
\end{figure}

\section{Desarrollo 1}

\lipsum[1-2]

\begin{quotation}
	\lipsum[1-2] (Fulano, 1999, p. 123)
\end{quotation}

\section{Desarrollo 2}

\lipsum[2-4]

\clearpage{\pagestyle{empty}\cleardoublepage}

\part{El presente}

\chapter{Otra introducción}

\lipsum[1-4]

\section{Otra sección}

\lipsum[1-3]

\subsection{Otra subsección}

\lipsum[4-6]

\clearpage{\pagestyle{empty}\cleardoublepage}

\chapter{Otro desarrollo}

\lipsum[1-4]

\section{Otro desarrollo 1}

\lipsum[1-2]

\section{Otro desarrollo 2}

\lipsum[2-4]

\clearpage{\pagestyle{empty}\cleardoublepage}

\part{El Futuro}


\chapter{La importancia del futuro}

\lipsum[1-10]

\section{Complemento}

\lipsum[2-3]


\chapter*{Bibliografía}

Apellido, N. (año). \emph{Título del libro}. País: Editorial.

Apellido, N. (año). \emph{Título del libro}. País: Editorial.

Apellido, N. (año). \emph{Título del libro}. País: Editorial.

Apellido, N. (año). \emph{Título del libro}. País: Editorial.

\clearpage{\pagestyle{empty}\cleardoublepage}

\section*{Otras fuentes}
\subsection*{Archivos}

Apellido, N. Nombre del documento. Año. Clasificación. Archivo, País.

Apellido, N. Nombre del documento. Año. Clasificación. Archivo, País.

Apellido, N. Nombre del documento. Año. Clasificación. Archivo, País.

\subsection*{Material visual}

Autor, N. (año). \emph{Obra} [Formato]. País.

Autor, N. (año). \emph{Obra} [Formato]. País.

Autor, N. (año). \emph{Obra} [Formato]. País.

Autor, N. (año). \emph{Obra} [Formato]. País.

\appendix


\chapter{La búsqueda del apéndice}

\lipsum[1-6]

\backmatter

\chapter{Epílogo}

\lipsum[1-6]

\end{document}